\documentclass{article}
\usepackage[margin=3cm,noheadfoot]{geometry}
\usepackage[lined, boxed, commentsnumbered]{algorithm2e}
\setcounter{secnumdepth}{3}
\title{The Network Layer}
\author{Chao Li}
\begin{document}
\maketitle
\section{Introduction}
\begin{itemize}
\item The primary role of the routers is to forward datagrams from
  input links to output links.
\subsection{Forwarding and Routing}
\item \emph{Forwarding}: When a packet arrives at a router's input
  link, the router must move the packet to the appropriate output
  link.
\item \emph{Routing}: The network layer must determine the route or
  path taken by packets as they flow from a sender to a receiver. The
  algorithms that calculate these paths are referred to as routing
  algorithms.
\item Forwarding refers to the router-local action of transferring a
  packet from an input link interface to the appropriate ouput link
  interface.
\item Routing refers to the network-wide process that determins the
  end-to-end paths that packets take from source to destination.
\item Every router has a forwarding table.
\item A router forwards a packet by examining the value of a field in
  the arriving packet's header, and then using this header value to
  index into the router's forwarding table.
\item Routing algorithm determines the values that are inserted into
  the router's forwarding tables..
\item A router receives routing protocol messages, which are used to
  configure its forwarding table.
\item \emph{Packet Switch}: A general packet-switching device that
  transfers a packet from input link interface to output link
  interface, according to the value in a field in the header of the
  packet.(Link-layer switch, Router).
\item \emph{Connection Setup} Routers along the chosen path to
  handshake with each other in order to set up state before the
  network-layer data packets within a given source-to-destination
  connection can begin to flow.
\subsection{Network Service Models}
\item The internet's network layer provides a single service, known as
  best-effort service.
\item \emph{Constant bit rate (CBR) ATM network service}. This is the
    first ATM service model to be standardized, reflecting early
    interest by the telephone companies in ATM and the suitability of
    CBR service for carrying real-time, constant bit rate audio and
    video traffic.
\item \emph{Avaliable bit rate (ABR) ATM network service} Slightly
  better than best effort service. Cells cannot be reordered, and a
  minimum cell transmission rate (MCR) is guaranteed to a
  connection. The ABR service can provide feedback to the sender to
  control how the sender adjust its rate between the MCR and an
  allowable peak cell rate.
\section{Virtual Circuit and Datagram Networks}
\item A network-layer connection service begins with handshaking
  between the source and destination hosts
\item a network-layer connectionless service does not have nay
  handshaking preliminaries.
\item Differences between transportation layer service with network
  layer service. 
\begin{center}
\begin{tabular}{|p{5cm}|p{5cm}|}
  \hline
  Transportation Layer Service & Network Layer Servifce \\ \hline
  process-to-process service   & host to host service \\ \hline
  -                            & provide only one type service, not
  both \\ \hline
  implemented at the edge of the network
                               & implemented in the routers as well as
                               the end systems \\ \hline
  
\end{tabular}
\end{center}
\subsection{Virtual-Circuit Networks}
\item a VC network consists of
\begin{enumerate}
\item a path between the source and destination hosts
\item VC numbers, one number for each link along the path
\item entries in the forwarding table in each router along the path
\end{enumerate}
\item each intervening router replaces the number in the packet's
  header
\item Reason why not just keep the same VC number on each of the links
  along its route
\begin{enumerate}
\item replacing the number from link to link reduces the length of the
  VC field in the packet header.
\item VC setup is considerably simplified by permitting a different VC
  number at each link along the path of the VC.
\end{enumerate}
\item Three identifiable phases in a VC:
\begin{enumerate}
\item VC setup
\item Data Transfer
\item VC tear down
\end{enumerate}
\item \emph{Signaling Messages}: The messages that the end systems
  send into the network to initiate or terminate a VC, and the
  messages passed between the routers to set up the VC (modify
  connection state in router tables).
\subsection{Datagram Networks}
\item In a \textbf{datagram network}, each time an end system wants to
  send a packet, it stamps the packet with the address of the
  destination end system and then pops the packet into the network.
\item Each router has a forwarding table that maps destination
addresses to link interfaces.
\item The router matches a prefix of the packet's destination address
  with the entries in the table.
\item When there are multiple matches, the router uses the
  \textbf{longest prefix matching rule}
\item Routers in datagram networks maintain no connection state
  information.
\item They maintain forwarding state information in their forwarding
  table.
\section{What's Inside a Router}
\item Four router components can be identified:
\begin{itemize}
\item Input ports
\begin{itemize}
\item it performs the physical layer function
\item it performs the link-layer functions
\item most crucial, it performs the lookup function.
\end{itemize}
\item Switching fabric: connects the router's input ports to its
  ourput ports.
\item Output ports: it stores packets received from the switching
  fabric and trnsmits them on the outgoing link.
\item Routing Processor: it executes the routing protocols; maintains
  routing tables and attached link state information; and computes the
  forwarding table for the router.
\end{itemize}
\subsection{Input Processing}
\item The forwarding table is computed and updated by the routing
  processor, with a shadow copy typically stored at each input port.
\item The forwarding table is copied from the routing processor to the
  line cards over a separate bus.
\item in some designs, a packet may be temporarily blocked from
  entering the switching fabric if packets from other input ports are
  currently using the fabric.
\item Many other actions must be taken:
\begin{enumerate}
\item physical- and link-layer processing
\item the packet's version number, ehcksum and time-to-live field must
  be checked and the latter two fields rewritten
\item counters used for network management must be updated.
\end{enumerate}
\subsection{Switching}
\item Three switching tecniques
\begin{itemize}
\item Switching via memory
\item Switching via a bus
\item Switching via an interconnection network
\end{itemize}
\subsection{Output Processing}
\item see page 327
\subsection{Where Does Queueing Occur?}
\item For many year, the rule of thunmb for buffer sizing was that the
  amount of bufferint $(B)$ shold be equal to an average round-trip
  time ($RTT$, say 250 msec) times the link capacity $(C)$.
\item Thus a $10 Gbps$ link with an RTT of $250 msec$ would need an
  aount of buffering equal to $B=RTT\cdot C=2.5 Gbits$ of buffers.
\item Recent suggest that when there are a large number of TCP flows
  $(N)$ passing through a link, the amount of buffering needed is
  $B=RTT\cdot C/ \sqrt{N}$.
\item \textbf{Packet Scheduler} at the ouput port must choose one
  packet for transmission.
\begin{itemize}
\item first-come-first-serverd
\item weighted fair queuing (WFQ).
\end{itemize}
\item Packet scheduling plays a crucial role in providing
  \textbf{quality-ofservice guarantees}
\item if out of memory, a decision must be made to either
\begin{itemize}
\item drop the arriving packet (apolicy known as \textbf{drop-tail})
\item remove one or more already-queued packets.
\end{itemize}
\item In some cases, it may be advantageous to drop a packet
  \emph{before} the buffer is full in order to provide a congestion
  signal to the sender.
\item A number of packet-dropping and -marking policies, collectively
  have become known as \textbf{active queue management (AQM)}
  algorithms
\item One of the most widely studied and implemented AQM algorithms is
  the \textbf{Random Early Detection (RED)} algorithm. see page 329
\item \textbf{head-of-the-line (HOL) blocking}: A queued packet in an
  input queue must wait for transfer through the fabric (even though
  its ouput port is free) because it is blocked by another packet at
  the head of the line.
\section{The Internet Protocol (IP): Forwarding and Addressing in the
  Internet}
\subsection{Datagram Format}
\item The maximum amount of data that a link-layer frame can carry is
  called the maximum transmission unit (MIU).
\item End systems rather than in network routers reassembles the
  datagram fragments.
\item IP header field used in datagram fragments:
  \emph{identification, flag} and \emph{fragmentation offset}
\item Page 337 has an example of the datagram fragmentation.
\subsection{IPv4 Addressing}
\item In IP terms, this network interconnecting three host interfaces
  and one router interface forms a \textbf{subnet}.
\item IP addressing assigns an address to this subnet: 223.1.1.0/24,
  where the /24 notation, somtimes known as a \textbf{subnet mask},
  indicates that theleftmost 24bits of the 23-bit quantity define the
  \emph{subnet address}.
\item page 340-341, subnet example.
\item The Interne's address assignment strategy is known as
  \textbf{Classless Interdomain Routing}(\textbf{CIDR}).
\item CIDR generalizes the notion of subnet addressing. As with subnet
  addressing, the 32-bit IP address is divided into two parts and
  again has the dotted-decimal form $a.b.c.d/x$, where $x$ indicates
  the number of bits in the first part of the address, are often
  referred to as the \textbf{prefix} of the IP address.
\item only these $x$ leading prefix bits are considered by routers
  outside the organization's network, which reduces the size of the
  forwarding table.
\item IP broadcast address 255.255.255.255
\subsubsection{Obtaining a Host Address: Dynamic Host Configuration
  Protocol}
\item DHCP allows a host to learn additional information, such as its
  subnet mask, the address of its first-hop router (often called the
  default gateway), and the address of its local DNS server.
\item For a newly arriving host, the DHCP protocol is a four-step
  process
\begin{enumerate}
\item \emph{DHCP server discovery}: client sends a DHCP discover
  message within a UDP packet to port 67.dest IP: 255.255.255.255,
  source IP: 0.0.0.0.
\item \emph{DHCP server offer}: Server responds to the client with a
  DHCP offer messsage. Dest IP: 255.255.255.255. Message contains the
  transaction ID, the proposed IP address for the client, the network
  mask, and an IP address lease time.
\item \emph{DHCP request}: client choose from among on or more server
  offers and respond to its selected offer with a DHCP request
  message.
\item \emph{DHCP ACK}
\end{enumerate}
\subsubsection{Network Address Translation (NAT)}
\item See Page 350-351 for details.
\subsection{Internet Control Message Protocol (ICMP)}
\item The network layer has three main components: The IP protocol,
  the Internet Routing Protocols, and ICMP.
\item ICMP is used by hosts and routers to communicate network-layer
  information to each other.
\item The most typical use of ICMP is for error reporting.
\item ICMP messages are carried inside IP datagrams.
\item ICMP messages have a type and a code field, and contain the
  header and the first 8 bytes of the IP datagram that caused the ICMP
  message to be generated in the first place
\subsection{IPv6}
\subsubsection{IPv6 Datagram Format}
\item \emph{Fragmentation/Reassembly}: IPv6 Does not allow for
  fragmentation and reassembly at intermediate routers; these
  operations can performed only by the source and destination.
\item If the router receive a packet which is too large, just send a
  `Packet Too Big' ICMP error message back to the sender. Sender then
  resend a smaller IP datagram. Speeding up IP forwarding within the
  network.
\item \emph{Header checksum} removed because of functinality exists in
  transport-layer and link-layer protocols.
\item Options field is also removed.
\subsubsection{Transitioning from IPv4 to IPv6}
\item \emph{dual-stack approach}: illustrated on page 360. Basically,
  if a IPv6 datagram encounter an IPv4 router, it will be rebuilt into
  a IPv4 datagram and hence some of the IPv6 information are lost.
\item \emph{Tunnerling}: Illustrated on page 361. Basically, the IPv6
  datagram is encapsuled into a payload (IPv4 header).
\subsection{A brief Foray into IP Security}
\item see page 362
\section{Routing Algorithms}
\item Typically a host is attached directly to one router, the
  \textbf{default router}.
\item \emph{global routing algorithm}: computes the least-cost path
  between a source and destination using complete, global knowledge
  about the network, e.g., \textbf{link-state (LS) algorithms}.
\item \emph{decentralized routing algorithm}, the calculation of the
  least-cost path is carried out in an iterative, distributed manner,
  e.g., \textbf{distance-vector (DV) algorithm}.
\item A second way to classify routing algorithms is according to
  whether they are static or dynamic.
\item A third way to classify routing algorithms is according to
  whether they are load-sensitive.
\item Today's Internet routing algorithms (such as RIP, OSPF and BGP)
  are load-insensitive. Links cost doesn't explicitly reflect its
  current level of congestion.
\subsection{The Link-State (LS) Routing Algorithm}
\begin{center}
\begin{algorithm}[H]
Initialization:\;
N' = {u}\;
\ForAll{nodes v} {
\eIf{v is a neighbor of u} {
D(v) = c(u, v)\;
}{
D(v) = $\infty$\;
}
}
\Repeat{N' = N} {
  find w not in N' such that D(w) is a minimum\;
  add w to N'\;
  update D(v) for each neighbor v of w and not in N': D(v) = min(
  D(v), D(w) + c(w, v) )\;
  /*new cost to v is either old cost to v or known least path cost to
  w plus cost from w to v*/\;
}


\caption{Link-State (LS) Algorithm for Source Node u}
\end{algorithm}
\end{center}
\subsection{The Distance-Vector (DV) Routing Algorithm}
\begin{center}
\begin{algorithm}
Initialization:\;
\ForAll{destinations y in N} {
$D_x(y) = c(x, y)$\;
}
\ForEach{neighbor w} {
	send distance vector $D_x = [D_x(y)]$: $y$ in $N$ to $w$\; 
}
\Repeat {forever}
{
	\textbf{wait} (until I see a link cost change to some neighbor $w$ or until I receive a distance vector from some neighbor $w$)\;
	\ForEach{$y$ in $N$}
	{
		$D_x(y) = min_v{c(x,v)+D_v(y)}$
	}
	\If {$D_x(y)$ changed for any destination $y$}
	{
		send distance vector $D_x=[D_x(y)]$: $y$ in $N$ to all neighbors
	}
}
\caption{Distance-Vector (DV) Algorithm}
\end{algorithm}
\end{center}
\subsubsection{DC Algorithm: Adding Poisoned Reverse}
\item if $z$ routes through $y$ to get to destination $x$, then $z$ will advertise to $y$ that its distance to x is infinity, that is, $z$ will advertise to $y$ that $D_z(x)=\infty$ (even though $z knows D_z(x)=5$ in truth). $z$ will continue telling this little white lie to $y$ as long as it routes to $x$ via $y$.
\subsubsection{A comparison of LS and DV Routing Algorithms}
\begin{center}
\begin{tabular}{| l |p{5cm}|p{5cm}|}
  \hline
   & LS Algorithm & DV Algorithm \\ \hline
  Message Complexity & $O(|N||E|)$   & Depand on many factors \\ \hline
  Speed of Convergence  & $O(N^2)$ & converge slowly \\ \hline
  Robustness & Routing calculation are somewhat separated, providing a degree of robustness & Erroneous propagate iteratively, which may cause significant problem. \\ \hline
\end{tabular}
$E$: the set of edges \\
$N$: the set of nodes \\
\end{center}
\subsubsection{Other Routing Algorithms}
\item \textbf{Circuit-Switched Routing Algorithms}: are of interest to packet-switched data networking in cases where per-link resources are to be reserved for each connection that is routed over the link. (From the telephony world).
\subsection{Hierarchical Routing}
\item Our previous view of network was simplized in the \emph{scale} and \emph{Administrative autonomy} two aspects.
\item Both of these problems can be solved by organizing routers into \textbf{autonomous systems (ASs)}, with each AS consisting of a group of routers that are typically under the same administrative control.
\item The routing algorithm running within an autonomous system is called an \textbf{autonomous system routing protocol}.
\item Routers that have the added task of being responsible for forwarding packets to destinations outside the AS are call \textbf{gateway routers}.
\item 
\end{itemize}

\end{document}
%%% Local Variables: 
%%% mode: latex
%%% TeX-master: t
%%% End: 
