\message{ !name(chapter4.tex)}\documentclass{article}
\title{The Network Layer}
\author{Chao Li}
\begin{document}

\message{ !name(chapter4.tex) !offset(0) }
\begin{itemize}
\item The primary role of the routers is to forward datagrams from
  input links to output links.
\subsection{Forwarding and Routing}
\item \emph{Forwarding}: When a packet arrives at a router's input
  link, the router must move the packet to the appropriate output
  link.
\item \emph{Routing}: The network layer must determine the route or
  path taken by packets as they flow from a sender to a receiver. The
  algorithms that calculate these paths are referred to as routing
  algorithms.
\item Forwarding refers to the router-local action of transferring a
  packet from an input link interface to the appropriate ouput link
  interface.
\item Routing refers to the network-wide process that determins the
  end-to-end paths that packets take from source to destination.
\item Every router has a forwarding table.
\item A router forwards a packet by examining the value of a field in
  the arriving packet's header, and then using this header value to
  index into the router's forwarding table.
\item Routing algorithm determines the values that are inserted into
  the router's forwarding tables..
\item A router receives routing protocol messages, which are used to
  configure its forwarding table.
\item \emph{Packet Switch}: A general packet-switching device that
  transfers a packet from input link interface to output link
  interface, according to the value in a field in the header of the
  packet.(Link-layer switch, Router).
\item \emph{Connection Setup} Routers along the chosen path to
  handshake with each other in order to set up state before the
  network-layer data packets within a given source-to-destination
  connection can begin to flow.
\subsection{Network Service Models}
\item The internet's network layer provides a single service, known as
  best-effort service.
\item \emph{Constant bit rate (CBR) ATM network service}. This is the
    first ATM service model to be standardized, reflecting early
    interest by the telephone companies in ATM and the suitability of
    CBR service for carrying real-time, constant bit rate audio and
    video traffic.
\item \emph{Avaliable bit rate (ABR) ATM network service} Slightly
  better than best effort service. Cells cannot be reordered, and a
  minimum cell transmission rate (MCR) is guaranteed to a
  connection. The ABR service can provide feedback to the sender to
  control how the sender adjust its rate between the MCR and an
  allowable peak cell rate.
\section{Virtual Circuit and Datagram Networks}
\item A network-layer connection service begins with handshaking
  between the source and destination hosts
\item a network-layer connectionless service does not have nay
  handshaking preliminaries.
\item Differences between transportation layer service with network
  layer service. 
\begin{center}
\begin{tabular}{|p{5cm}|p{5cm}|}
  \hline
  Transportation Layer Service & Network Layer Servifce \\ \hline
  process-to-process service   & host to host service \\ \hline
  -                            & provide only one type service, not
  both \\ \hline
  implemented at the edge of the network
                               & implemented in the routers as well as
                               the end systems \\ \hline
  
\end{tabular}
\end{center}
\subsection{Virtual-Circuit Networks}
\item a VC network consists of
\begin{enumerate}
\item a path between the source and destination hosts
\item VC numbers, one number for each link along the path
\item entries in the forwarding table in each router along the path
\end{enumerate}
\item each intervening router replaces the number in the packet's
  header
\item Reason why not just keep the same VC number on each of the links
  along its route
\begin{enumerate}
\item replacing the number from link to link reduces the length of the
  VC field in the packet header.
\item VC setup is considerably simplified by permitting a different VC
  number at each link along the path of the VC.
\end{enumerate}
\item Three identifiable phases in a VC:
\begin{enumerate}
\item VC setup
\item Data Transfer
\item VC tear down
\end{enumerate}
\item \emph{Signaling Messages}: The messages that the end systems
  send into the network to initiate or terminate a VC, and the
  messages passed between the routers to set up the VC (modify
  connection state in router tables).
\subsection{Datagram Networks}
\item In a \textbf{datagram network}, each time an end system wants to
  send a packet, it stamps the packet with the address of the
  destination end system and then pops the packet into the network.
\item Each router has a forwarding table that maps destination
addresses to link interfaces.
\item The router matches a prefix of the packet's destination address
  with the entries in the table.
\item When there are multiple matches, the router uses the
  \textbf{longest prefix matching rule}
\item Routers in datagram networks maintain no connection state
  information.
\item They maintain forwarding state information in their forwarding
  table.
\subsection{What's Inside a Router}
\item Four router components can be identified:
\begin{itemize}
\item Input ports
\begin{itemize}
\item it performs the physical layer function
\item it performs the link-layer functions
\item most crucial, it performs the lookup function.
\end{itemize}
\item Switching fabric: connects the router's input ports to its
  ourput ports.
\item Output ports: it stores packets received from the switching
  fabric and trnsmits them on the outgoing link.
\item Routing Processor: it executes the routing protocols; maintains
  routing tables and attached link state information; and computes the
  forwarding table for the router.
\end{itemize}
\subsection{Input Processing}
\item The forwarding table is computed and updated by the routing
  processor, with a shadow copy typically stored at each input port.
\item The forwarding table is copied from the routing processor to the
  line cards over a separate bus.
\item in some designs, a packet may be temporarily blocked from
  entering the switching fabric if packets from other input ports are
  currently using the fabric.
\item Many other actions must be taken:
\begin{enumerate}
\item physical- and link-layer processing
\item the packet's version number, ehcksum and time-to-live field must
  be checked and the latter two fields rewritten
\item counters used for network management must be updated.
\end{enumerate}
\subsection{Switching}
\item Three switching tecniques
\begin{itemize}
\item Switching via memory
\item Switching via a bus
\item Switching via an interconnection network
\end{itemize}
\subsection{Output Processing}
\item see page 327
\subsection{Where Does Queueing Occur?}
\item For many year, the rule of thunmb for buffer sizing was that the
  amount of bufferint $(B)$ shold be equal to an average round-trip
  time ($RTT$, say 250 msec) times the link capacity $(C)$.
\item Thus a $10 Gbps$ link with an RTT of $250 msec$ would need an
  aount of buffering equal to $B=RTT\cdot C=2.5 Gbits$ of buffers.
\item Recent suggest that when there are a large number of TCP flows
  $(N)$ passing through a link, the amount of buffering needed is
  $B=RTT\cdot C/\sqrt{N}$.
\item \textbf{Packet Scheduler} at the ouput port must choose one
  packet for transmission.
\begin{itemize}
\item first-come-first-serverd
\item weighted fair queuing (WFQ).
\end{itemize}
\item Packet scheduling plays a crucial role in providing
  \textbf{quality-ofservice guarantees}
\item if out of memory, a decision must be made to either
\begin{itemize}
\item drop the arriving packet (apolicy known as \textbf{drop-tail)
\item remove one or more already-queued packets.
\end{itemize}
\item In some cases, it may be advantageous to drop a packet
  \emph{before} the buffer is full in order to provide a congestion
  signal to the sender.
\item A number of packet-dropping and -marking policies, collectively
  have become known as \textbf{active queue management (AQM)}
  algorithms
\item One of the most widely studied and implemented AQM algorithms is
  the \textbf{Random Early Detection (RED)} algorithm. see page 329
\item \textbf{head-of-the-line (HOL) blocking}: A queued packet in an
  input queue must wait for transfer through the fabric (even though
  its ouput port is free) because it is blocked by another packet at
  the head of the line.
\section{The Internet Protocol (IP): Forwarding and Addressing in the Internet}
\end{itemize}
\message{ !name(chapter4.tex) !offset(-1) }

\end{document}
%%% Local Variables: 
%%% mode: latex
%%% TeX-master: t
%%% End: 
